\documentclass[10pt]{article}
\usepackage{hyperref}
\usepackage{xspace}
\newcommand{\zlib}{\texttt{zlib}\xspace}

\title{Final Project Report:\\ Compressed Checkpoint Index \& Parallel GZIP
Reader}
\author{Josh Tang, Aaron Ortwein, Erik Rye}

\begin{document}
\maketitle

\section{Related Work}

Kerbiriou and Chikhi examined parallel decompression of FASTQ gzip files to
speed up genomic algorithms that require FASTQ
input~\cite{kerbiriou2019parallel}. They do not build an index over the FASTQ
file by making a pass over it as we attempt to do, but rather attempt to
reconstruct data from back-references discovered by reading forward from a
random entry point.

\texttt{zlib}~\cite{zlib} is a C library written by Mark Adler and others that
is able to create and read gzip files. This library will be important for our
project as it exposes the underlying mechanics for inflating gzip files.

\texttt{pyfastx}~\cite{pyfastx} is a Python C extension that enables random
reads from FASTA and FASTQ. The functionality that it exposes is similar to our
overall goal and will be a useful source of information and point of comparison.

\texttt{indexed\_gzip}~\cite{indexedgzip} is a project by Paul McCarthy to
create an index over compressed NIFTI image files, which use gzip compression by
default. This will be another useful point of comparison in a tool that needs to
accomplish a similar task as us.

\begin{table}[ht]
    \caption{Four sources of FASTQ data were used in our study. The FASTQ files
    were gzip compressed for our index-building and parallel reading
    experiments.}
\begin{tabular}{r|l|r|r}
\multicolumn{1}{c}{\textbf{\begin{tabular}[c]{@{}c@{}}Sequence Read\\
Identifier\end{tabular}}} & \multicolumn{1}{c}{\textbf{Source}} &
    \multicolumn{1}{c}{\textbf{Sequence Reads}} & \multicolumn{1}{c}{\textbf{\begin{tabular}[c]{@{}c@{}}FASTQ GZ \\ Size (MB)\end{tabular}}} \\
\hline\\
SRR3295681                                                                                      & Salmonella enterica                 & 959,879                                     & 205                                                                                        \\
SRR2121685                                                                                      & Mus musculus                        & 27,928,438                                  & 2,078                                                                                      \\
SRR925811                                                                                       & Homo sapiens                        & 53,265,409                                  & 3,349                                                                                      \\
SRR925816                                                                                       & Homo sapiens                        & 71,504,007                                  & 5,046
\end{tabular}
    \label{tab:source}
\end{table}

\section{Data Sources}

We used four reference FASTQ sequence reads from the Sequence Read
Archive~\cite{SRA} for our testing and analysis. Table~\ref{tab:source} is a
tabular depiction of the data sets that we used for this project. We chose these
files because they represented a wide range of numbers of sequence reads, which
translates into a wide range of gzip file sizes for our tool to contend with.
Indeed, the largest FASTQ gzip file we consider is approximately 25 times as
large as the shortest.

\section{Index Creation Tool \& Methodology}

We implemented our solution in C in order to take advantage of the library
\zlib~\cite{zlib}. \zlib is a C library written by Mark Adler

\section{Parallel Reading}

\subsection{Methodology}

\subsection{Results}

\section{Link to your workspace}

Our work is located here:
\url{https://github.com/Josh-Tang112/CMSC701_Final_Project}

\bibliographystyle{plain}
\bibliography{refs.bib}


\end{document}
