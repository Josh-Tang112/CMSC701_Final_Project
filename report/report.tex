\documentclass[unnumsec,webpdf,contemporary,large]{oup-authoring-template}
\usepackage{hyperref}
\usepackage{xspace}
\usepackage{graphicx}
\usepackage{subfigure}
\newcommand{\zlib}{\texttt{zlib}\xspace}
\newcommand{\zran}{\texttt{zran}\xspace}
\newcommand{\ibuilder}{\texttt{index\_builder}\xspace}
\newcommand{\ireader}{\texttt{index\_reader}\xspace}
\newcommand{\gzip}{gzip\xspace}
\newcommand{\eg}{\emph{e.g.,}\xspace}

\begin{document}

\journaltitle{CMSC701 Final Project}
\DOI{}
\copyrightyear{2023}
\pubyear{2023}
\appnotes{Final Project}

\firstpage{1}

%\subtitle{Subject Section}

\title[Compressed Checkpoint Index]{Final Project Report: Compressed Checkpoint
Index \& Parallel \gzip Reader}

\author[1]{Erik Rye}
\author[1]{Jiaxi Tang}
\author[1]{Aaron Ortwein}

%\authormark{Author Name et al.}

\address[1]{\orgdiv{Computer Science Department}, \orgname{University of
Maryland}, \orgaddress{\street{8125 Paint Branch Dr, College Park},
\postcode{20742}, \state{MD}, \country{USA}}}

%\corresp[$\ast$]{Corresponding author. \href{email:email-id.com}{email-id.com}}

%\received{Date}{0}{Year}
%\revised{Date}{0}{Year}
%\accepted{Date}{0}{Year}

\abstract{We devise and describe a tool for creating an index over a \gzip FASTQ
file. Unlike standard index-creation tools, our tool and the indices that it
creates incorporate a user-specified number of sequence reads that should be
split into each index. This allows a different tool that reads a FASTQ \gzip
file with our pre-built index over it to efficiently process the sequence reads
\emph{in parallel}, rather than needing to process the entire file serially.
This can result in a dramatic speedup for some applications, as we demonstrate
with the simple use case of reading the entire FASTQ file into memory and
writing the decompressed result back to a file.}

\keywords{FASTQ, \gzip, Compression, Parallel Processing}

% \boxedtext{
% \begin{itemize}
% \item Key boxed text here.
% \item Key boxed text here.
% \item Key boxed text here.
% \end{itemize}}

\maketitle

\section{Introduction}
Computer processors are now capable of running hundreds of threads of execution simultaneously in parallel. With severe physical limits on clock speed, future architectures will likely support more simultaneous threads rather than faster individual cores ~\cite{intropaper}. These advances provides programmers new way to speed up their programs. However, simply using more threads to exexute parts of the program does not guarantee speedup and may very well speed down than speed up. In fact, it is not uncommon for a program’s overall throughput to decrease when thread count grows large enough ~\cite{intropaper}. So, to speedup genomics software with multiple number of threads, programmers should deliberate on how they should structure the entire program and how many threads they should use so that the overhead of multithreading will not outweigh the speedup it brings.

Here we try to solve the problem of decompressing a gzip compressed FASTQ files in parallel by building indices over it. 
This startegies of building indices can scale to hundreds of threads and make it easier for our program to be part of the pipeline of other multithreading or multiprocessing genomics tools. 
\subsection{Challenge of Multithreading}
Some of the challenges multithreading will pose can be seen in Figure 1. Figure 1 shows how bad multithreading can be if handled incorrectly. We can see that if a thread needs to read from or write to a file, it needs exclusive right to the file so that the file won't be changed when it's reading it. This is usually done by using locks. Unfortunately, even though multiple threads can have read lock of the same file, operating system does not really allows them to read the file simultaneously. That means if multiple thread is trying to read a file, most of them cannot do anything and has to wait for that one thread to complete first. The same applies to writing to a file as well. Therefore, it's important to structure the program and choose parameters wisely so that the time spent waiting for each thread is minimum. 
\begin{figure}[H]
    \includegraphics[width=\linewidth]{figs/multithread.jpeg}
    \label{fig:multithread}
    \caption{This figure shows 4 threads running simultaneously . Time progress from top to bottom. Gray area represents time spent waiting instead of operating. Black area represents threads doing operations that require exclusive rights to certain resources. ~\cite{intropaper}}
\end{figure}


\section{Related Work}

Building an index over a \gzip file has been a focus of several recent projects,
many of which take advantage of \texttt{zlib}~\cite{zlib}, a C libray originally
authored by Jean-loup Gailly and Mark Adler for the purpose of creating and reading
\gzip files. The canonical example of building an index for random access into a \gzip 
file is zran.c, an extension of \texttt{zlib} written by Mark Adler. zran.c decompresses 
the \gzip file block by block, and at each block boundary, it will create an index access 
point if it has consumed a user-configurable number of bytes in the uncompressed data. 
Each access point stores the block offset in both the compressed and uncompressed data, 
as well as a 32 KB window of uncompressed data immediately preceding the block needed 
to resolve back references during decompression. Then, given an offset in the uncompressed 
file, zran.c will find the closest access point that occurs at or before the offset, 
initialize the decoder state with the 32 KB window, decompress and discard data until
reaching the offset, then continue to decompress a desired number of bytes before returning
the result to the user.

At least two subsequent projects directly utilize zran.c for building an index over 
compressed files for which \gzip is the most common compression standard.
\texttt{indexed\_gzip}~\cite{indexedgzip} is a project by Paul McCarthy designed
to create an index over compressed NIFTI image files. Because zran.c can build an
index over any \gzip file independent of the format of the uncompressed data, the
Python C extension \texttt{pyfastx}~\cite{pyfastx} uses \texttt{indexed\_gzip} to
enable random reads from compressed FASTA and FASTQ files.

Kerbiriou and Chikhi take a different approach to decompressing FASTQ \gzip files
at arbitrary locations, doing so in parallel and without an index~\cite{kerbiriou2019parallel}.
Their decompression algorithm takes two passes over the FASTQ \gzip file. The first
pass divides the \gzip file into chunks, each of which will be decompressed by a single
thread. Without the 32 KB context preceding each block, only the thread handling 
decompression of the first chunk will be able to completely resolve back references;
every back reference points somewhere within the first chunk. Because all other chunks
will have back pointers referring to previous chunks, the threads handling those chunks
will mark the pointers as unresolved and inflate as much of the chunk as possible. The 
second pass provides the last 32 KB of the inflated chunks to threads decompressing 
later chunks so that they can resolve remaining back pointers.

% \texttt{libdeflate}~\cite{libdeflate} is a state-of-art highly optimized single-threaded open-source library used to compress and decompress files in DEFLATE/gzip/zlib format. This will serve as a great benchmark for our program.


\section{Methodology}
We implemented our solution in C in order to take advantage of the library
\zlib~\cite{zlib}. \zlib is a C library written and maintained by Jean-loup
Gailly and Mark Adler since 1995.
The software tools we developed for our project consist of an \emph{index
building} component, which creates the sequence-aware \gzip index, and an
\emph{index reading} component, which we use to validate the utility of the
index building component by using the indices to read a compressed gzip file in
parallel and write it out to a file, which we then compare against the
uncompressed original FASTQ to verify our program's correctness. We first detail
the thought process and inspriation behind our approach in
Subsection~\ref{sec:zran}, document the design of our index builder in
Subsection~\ref{sec:ibuilder}, and discuss the design of the index reader and validation
tool in Subsection~\ref{sec:ireader}.

\subsection{\zran}
\label{sec:zran}

The starting point for our index creation utility is a tool called
\zran~\cite{zran}. \zran
is a single-file demonstration of how to build an index over a \gzip file
written by Mark Adler, co-creator and maintainer of \zlib. It is included with
\zlib. \zran is designed to allow for random access within a \gzip file. \zran
accomplishes this in two steps. 

First, \zran completes a full read through the \gzip file. At each \gzip block boundary, which is defined as a literal with value 256 in DEFLATE, \zran will add an access point if \zran has consumed more than SPAN number of bytes, specified by the user, and such block is not the last block. 

Then, given an offset to the uncompressed file, \zran will find an access point right before the offset and skip to the DEFLATE block that offset is in while deflating the skipped block to maintain the context needed for deflate. After the block offset is in is fetched and context for it is built, \zran will read in desired number of uncompressed bytes, decompressed them, and return the result to the user. 

\subsection{\ibuilder}
\label{sec:ibuilder}
We built our \ibuilder tool around the idea of access point in \zran. By the
purpose of this program, each index cannot end in the middle of a read because
that will defeat the purpose of making this program part of the pipeline of
other multithreading or multiprocessing genomics tools. But, because \zran was
not designed for FASTQ file or genomics files in general, we have to modify
\zran to add more information to our index. Specifically, we wish to create
indices only for \gzip blocks that contain the start of a sequence chunk of
reads (\eg 10,000), a number that can be specified by the user. Therefore, we
need to restrict the program to only create an index at the start of a block
that contains the start of a sequence read chunk.

This, however, creates a new set of problems. The start of each DEFLATE block
contains information needed to decompress the block. If the index is not on
block boundary, deflate algorithm will fail to execute unless we store the
Huffman code tree at the start of each block alongside the index. We decided
against this approach and chose to store the index of start of the corresponding
block for each index and how many uncompressed bytes we need to skip instead. 

The reason for this is simple. If the index is at the middle of a block, we have to tell deflate algorithm how many bits in the byte the index points it should skip. In other words, we need to know the valid starting point in the byte which standard \zlib~\cite{zlib} will not provide. That means we have to write our own deflate algorithm, which is impossible due to time constraints and limitation of our knowledge about implementation of the deflate algorithm. 

The downside of this approach is quite limited considering two facts at hand.
First, we won't create an index until both the block and the read at the end of
the block end, which means in the worst case scenario, we only have
one-read-length amount of bytes of overhead per index. Second, each read is
short in FASTQ. If we can control the number of index points, we can make the
overhead of our approach negligible.  

Our \ibuilder implementation is written in C and uses \zlib; indeed, it was
created by modifying \zran directly. Like \zran, the end of each DEFLATE block,
\ibuilder adds a new \texttt{access\_point} index entry to a list of index
points that it maintains while reading the \gzip FASTQ file. Unlike \zran, we
also parse the decoded byte buffer to count the line numbers within the FASTQ
file as we decompress it. We do this because the FASTQ format specifies that
every fourth line is the start of a new sequence. Therefore, by maintaining
state on what line number we are on as we inflate the compressed file, we can
track which sequence are on by incrementing a counter every fourth line. 

\ibuilder in addition to the required \gzip file parameter, \ibuilder accepts an
optional \texttt{-c CHUNKSIZE} parameter that allows the user to specify the
number of sequence reads in a ``chunk.'' When the \ibuilder has read
\texttt{CHUNKSIZE} sequences (by default 10,000), it denotes which DEFLATE block
number it is currently in, and the number of bytes from the start of the current
block the sequence chunk begins at. It adds this information to another linked
list in order to output this information when the program completes. In this
manner, \ibuilder creates an index over the \gzip FASTQ file, annotating both
the start of \gzip DEFLATE blocks and the start of sequence chunks within those
blocks. When the program has finished reading the file, it outputs two index
files -- \texttt{output.idx} and \texttt{output.seq-idx}. The former contains
information allowing a reader to begin decompression at the beginning of a
DEFLATE block -- the number of bytes the block starts from the beginning of the
compressed file, what byte number that point is in the uncompressed data, and
32kB worth of data immediately preceding it in order to initialize the
decompression state. The latter contains information that allows the reader to
seek from the beginning of a DEFLATE block to the beginning of a sequence chunk
-- namely, which DEFLATE block number the sequence chunk occurs within, and the
byte offset, which allows a reader to calculate how much data exists between the
start of the block and the start of the chunk, which must be discarded.

\subsection{\ireader}
\label{sec:ireader}
Like \zran, our \ireader will skip to the desired DEFLATE block while
maintaining the necessary context for deflate algorithm. After it fetches the
block and creates the context, it will read all the bytes from this index up to
the next index, decompress them, and return the result to user.

\begin{table*}[ht]
    \centering
    \caption{Four sources of FASTQ data were used in our study. The FASTQ files
    were \gzip compressed for our index-building and parallel reading
    experiments.}
\begin{tabular}{r|l|r|r}
\multicolumn{1}{c}{\textbf{\begin{tabular}[c]{@{}c@{}}Sequence Read\\
Identifier\end{tabular}}} & \multicolumn{1}{c}{\textbf{Source}} &
    \multicolumn{1}{c}{\textbf{Sequence Reads}} & \multicolumn{1}{c}{\textbf{\begin{tabular}[c]{@{}c@{}}FASTQ GZ \\ Size (MB)\end{tabular}}} \\
\hline\\
SRR3295681 (Small)& Salmonella enterica & 959,879 & 205\\
SRR2121685 (Medium) & Mus musculus & 27,928,438 & 2,078\\
SRR925811  (Large) & Homo sapiens & 53,265,409 & 3,349 \\
SRR925816 (XL) & Homo sapiens & 71,504,007 & 5,046
\end{tabular}
    \label{tab:source}
\end{table*}

\subsection{Data Sources}

We used four reference FASTQ sequence reads from the Sequence Read
Archive~\cite{SRA} for our testing and analysis. Table~\ref{tab:source} is a
tabular depiction of the data sets that we used for this project. We chose these
files because they represented a wide range of numbers of sequence reads, which
translates into a wide range of \gzip file sizes for our tool to contend with.
Indeed, the largest FASTQ \gzip file we consider is approximately 25 times as
large as the shortest.


\section{Results}
\label{sec:results}

In this section we discuss the results of using our \ibuilder to create index
files over variously-sized \gzip FASTQ files, and then turn to the results of
running \ireader on those index files.

\subsection{Building Indices}
\label{sec:buildresults}


Using the \gzip FASTQ samples we described in Table~\ref{tab:source}, we run
\ibuilder over each. For each sample, we record the time it takes to build the
index files, the number of indices that each file contains, and the number of
bytes in each of the resulting indexes. 

While our \ibuilder implementation
writes separate files that contain \gzip DEFLATE block boundary information and



\subsection{Parallel Reading}
\label{sec:readresults}

\begin{figure}[h]
    \includegraphics[width=\linewidth]{figs/cores.pdf}
    \label{fig:cores}
    \caption{Time to read differently-sized FASTQ \gzip files using a pre-built
    index file and different numbers of threads. Time compasses total program
    execution, including reading the index files, parallel reading the \gzip
    file with $N$ threads, and writing the reconstituted file to disk. Time for
    \texttt{zcat} to read the file and write it to disk is shown for comparison
    using a dotted line (\texttt{zcat} execution is single threaded). See
    Table~\ref{tab:source} for descriptions of small, medium, large, and XL.}
\end{figure}

Using the index files we generated for each 



\section{Conclusion and Future Work}

Our work is located here:
\url{https://github.com/Josh-Tang112/CMSC701_Final_Project}



\bibliographystyle{plain}
\bibliography{refs.bib}


\end{document}
