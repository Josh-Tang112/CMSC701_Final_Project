\section{Methodology}
We implemented our solution in C in order to take advantage of the library
\zlib~\cite{zlib}. \zlib is a C library written and maintained by Jean-loup
Gailly and Mark Adler since 1995.
The software tools we developed for our project consist of an \emph{index
building} component, which creates the sequence-aware \gzip index, and an
\emph{index reading} component, which we use to validate the utility of the
index building component by using the indices to read a compressed gzip file in
parallel and write it out to a file, which we then compare against the
uncompressed original FASTQ to verify our program's correctness. We first detail
the thought process and inspriation behind our approach in
Subsection~\ref{sec:zran}, document the design of our index builder in
Subsection~\ref{sec:ibuilder}, and discuss the design of the index reader and validation
tool in Subsection~\ref{sec:ireader}.

\subsection{\zran}
\label{sec:zran}

The starting point for our index creation utility is a tool called
\zran~\cite{zran}. \zran
is a single-file demonstration of how to build an index over a \gzip file
written by Mark Adler, the co-creator and maintainer of \zlib. It is included with
\zlib. \zran is designed to allow for random access within a \gzip file. \zran
accomplishes this in two steps. 

First, \zran completes a full read through the \gzip file. At each \gzip block boundary, which is defined as a literal with value 256 in DEFLATE, \zran will add an access point if \zran has consumed more than SPAN number of bytes, specified by the user, and such block is not the last block. 

Then, given an offset to the uncompressed file, \zran will find an access point right before the offset and skip to the DEFLATE block that offset is in while deflating the skipped block to maintain the context needed for deflate. After the block offset is in is fetched and context for it is built, \zran will read in desired number of uncompressed bytes, decompressed them, and return the result to the user. 

\subsection{\ibuilder}
\label{sec:ibuilder}
We built our \ibuilder tool around the idea of access point indices from \zran.
To accomplish our project's purpose, however,
these indices cannot end in the middle of a read because
that will defeat the purpose of making this program part of the pipeline of
other multithreading or multiprocessing genomics tools. Because \zran was
not designed for FASTQ file or genomics files in general, we have to modify
\zran to add more information to our index. Specifically, we wish to create
indices only for \gzip blocks that contain the start of a sequence chunk of
reads (\eg 10,000), a number that can be specified by the user. Therefore, we
need to restrict the program to only create indices at the start of blocks
that contains the start of a sequence read chunk.

This, however, creates a new set of problems. The start of each DEFLATE block
contains information needed to decompress the block. If the index is not on
block boundary, inflate algorithm will fail to execute unless we store the
Huffman code tree at the start of each block alongside the index. We decided
against this approach and chose to store the index of start of the corresponding
block for each index and how many uncompressed bytes we need to skip instead. 

The reason for this is simple. If the index is at the middle of a block, we have to tell inflate algorithm how many bits in the byte the index points it should skip. In other words, we need to know the valid starting point in the byte which standard \zlib~\cite{zlib} will not provide. That means we have to write our own inflate algorithm, which is impossible due to time constraints and limitation of our knowledge about implementation of the inflate algorithm. 

The downside of this approach is quite limited considering two facts at hand.
First, we won't create an index until both the block and the read at the end of
the block end, which means in the worst case scenario, we only have
one-read-length amount of bytes of overhead per index. Second, each read is
short in FASTQ. If we can control the number of index points, we can make the
overhead of our approach negligible.  

Our \ibuilder implementation is written in C and uses \zlib; indeed, it was
created by modifying \zran directly. Like \zran, the end of each DEFLATE block,
\ibuilder adds a new \texttt{access\_point} index entry to a list of index
points that it maintains while reading the \gzip FASTQ file. Unlike \zran, we
also parse the decoded byte buffer to count the line numbers within the FASTQ
file as we decompress it. We do this because the FASTQ format specifies that
every fourth line is the start of a new sequence. Therefore, by maintaining
state on what line number we are decoding as we inflate the compressed file, we can
track which sequence we are on by incrementing a counter every fourth line. 

In addition to the required \gzip file parameter, \ibuilder accepts an
optional \texttt{-c CHUNKSIZE} parameter that allows the user to specify the
number of sequence reads in a ``chunk.'' When the \ibuilder has read
\texttt{CHUNKSIZE} sequences (by default 10,000), it denotes which DEFLATE block
number it is currently in, and the number of bytes from the start of the current
block the sequence chunk begins at. It adds this information to another linked
list in order to output this information when the program completes. In this
manner, \ibuilder creates an index over the \gzip FASTQ file, annotating both
the start of \gzip DEFLATE blocks and the start of sequence chunks within those
blocks. 

When the \ibuilder program has finished reading the file, it outputs two index
files -- \texttt{output.idx} and \texttt{output.seq-idx}. The former contains
information allowing a reader to begin decompression at the beginning of a
DEFLATE block -- the number of bytes the block starts from the beginning of the
compressed file, what byte number that point is in the uncompressed data, and
32kB worth of data immediately preceding it in order to initialize the
decompression state. The latter contains information that allows the reader to
seek from the beginning of a DEFLATE block to the beginning of a sequence chunk
-- namely, which DEFLATE block number the sequence chunk occurs within, and the
byte offset, which allows a reader to calculate how much data exists between the
start of the block and the start of the chunk, which must be discarded.

\subsection{\ireader}
\label{sec:ireader}

Like \zran, our \ireader will skip to the desired DEFLATE block while
maintaining the necessary context for inflate algorithm. After it fetches the
block and creates the context, it will read all the bytes from this index up to
the next index, decompress them, and return the result to user. In our initial
prototype, we have \ireader write the decompressed FASTQ file to disk. While
this task is not specifically related to genomics, it allows us to make easy
benchmarking comparisons to single-threaded tools that accomplish the same task,
such as \texttt{zcat}, which simply decompresses a \gzip file. It also allows
us to easily compare our parallel, FASTQ-aware \ireader implementation's output
with the decompressed FASTQ file to validate the correctness of our
implementation.

\ireader has three mandatory arguments and one optional argument. A user must
provide the \texttt{.idx} and \texttt{.seq-idx} files produced by \ibuilder, as
well as the \gzip file that they were built over. The user may also specify the
number of threads they wish to use to read the file in parallel; the number of
threads is set to 4 by default. The allowed thread number is bounded between 1
and 16.

\begin{table*}[ht]
    \centering
    \caption{Four sources of FASTQ data were used in our study. The FASTQ files
    were \gzip compressed for our index-building and parallel reading
    experiments.}
\begin{tabular}{r|l|r|r}
\multicolumn{1}{c}{\textbf{\begin{tabular}[c]{@{}c@{}}Sequence Read\\
Identifier\end{tabular}}} & \multicolumn{1}{c}{\textbf{Source}} &
    \multicolumn{1}{c}{\textbf{Sequence Reads}} &
    \multicolumn{1}{c}{\textbf{\begin{tabular}[c]{@{}c@{}}FASTQ GZ \\ Size (MB)\end{tabular}}} \\
\hline\\
SRR3295681 (Small)& Salmonella enterica & 959,879 & 205\\
SRR2121685 (Medium) & Mus musculus & 27,928,438 & 2,078\\
SRR925811  (Large) & Homo sapiens & 53,265,409 & 3,349 \\
SRR925816 (XL) & Homo sapiens & 71,504,007 & 5,046
\end{tabular}
    \label{tab:source}
\end{table*}

\subsection{Data Sources}

We used four reference FASTQ sequence reads from the Sequence Read
Archive~\cite{SRA} for our testing and analysis. Table~\ref{tab:source} is a
tabular depiction of the data sets that we used for this project. We chose these
files because they represented a wide range of numbers of sequence reads, which
translates into a wide range of \gzip file sizes for our tool to contend with.
Indeed, the largest FASTQ \gzip file we consider is approximately 25 times as
large as the shortest.

