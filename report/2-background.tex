\section{Related Work}

Building an index over a \gzip file has been a focus of several recent projects,
many of which take advantage of \texttt{zlib}~\cite{zlib}, a C libray originally
authored by Jean-loup Gailly and Mark Adler for the purpose of creating and reading
\gzip files. The canonical example of building an index for random access into a \gzip 
file is zran.c, an extension of \texttt{zlib} written by Mark Adler. zran.c decompresses 
the \gzip file block by block, and at each block boundary, it will create an index access 
point if it has consumed a user-configurable number of bytes in the uncompressed data. 
Each access point stores the block offset in both the compressed and uncompressed data, 
as well as a 32 KB window of uncompressed data immediately preceding the block needed 
to resolve back references during decompression. Then, given an offset in the uncompressed 
file, zran.c will find the closest access point that occurs at or before the offset, 
initialize the decoder state with the 32 KB window, decompress and discard data until
reaching the offset, then continue to decompress a desired number of bytes before returning
the result to the user.

At least two subsequent projects directly utilize zran.c for building an index over 
compressed files for which \gzip is the most common compression standard.
\texttt{indexed\_gzip}~\cite{indexedgzip} is a project by Paul McCarthy designed
to create an index over compressed NIFTI image files. Because zran.c can build an
index over any \gzip file independent of the format of the uncompressed data, the
Python C extension \texttt{pyfastx}~\cite{pyfastx} uses \texttt{indexed\_gzip} to
enable random reads from compressed FASTA and FASTQ files.

Kerbiriou and Chikhi take a different approach to decompressing FASTQ \gzip files
at arbitrary locations, doing so in parallel and without an index~\cite{kerbiriou2019parallel}.
Their decompression algorithm takes two passes over the FASTQ \gzip file. The first
pass divides the \gzip file into chunks, each of which will be decompressed by a single
thread. Without the 32 KB context preceding each block, only the thread handling 
decompression of the first chunk will be able to completely resolve back references;
every back reference points somewhere within the first chunk. Because all other chunks
will have back pointers referring to previous chunks, the threads handling those chunks
will mark the pointers as unresolved and inflate as much of the chunk as possible. The 
second pass provides the last 32 KB of the inflated chunks to threads decompressing 
later chunks so that they can resolve remaining back pointers.

% \texttt{libdeflate}~\cite{libdeflate} is a state-of-art highly optimized single-threaded open-source library used to compress and decompress files in DEFLATE/gzip/zlib format. This will serve as a great benchmark for our program.

