\documentclass[10pt]{article}
\usepackage{hyperref}
\title{Intermediate Progress Report:\\ Compressed Checkpoint Index}
\author{Josh Tang, Aaron Ortwein, Erik Rye}

\begin{document}
\maketitle
\section{Related work}

Kerbiriou and Chikhi examined parallel decompression of FASTQ gzip files to
speed up genomic algorithms that require FASTQ
input~\cite{kerbiriou2019parallel}. They do not build an index over the FASTQ
file by making a pass over it as we attempt to do, but rather attempt to
reconstruct data from back-references discovered by reading forward from a
random entry point.

\section{Plan of Work}

\begin{itemize}

\item Determine what library to use. Initial research suggests this is zlib.
\item Determine when index entries can be made 
\item Determine what information needs to be stored 
\item Determine what format to store the index data in 
\item Create a library that exposes the ``read starting at sequence number
        X'' given an index logic.
\item Compare the time to read various sizes of gzipped FASTQ files using a
    number of threads with the index file and our library vs a sequential
        reader.
\end{itemize}

\section{Key Questions}

\begin{enumerate}
\item Can index entries be made at any  byte indices, or must they be created at
    a specific point in the gzip file (e.g., at a block boundary?)
\item what is the data structure in zlib that controls
    compression/decompression, and what struct members/other state is required
        in order to reinitialize that structure?
\item Research suggests the zlib state requires a circular buffer of 32KB; how
    to best minimize index size?
\end{enumerate}

\section{Link to your workspace}

Our work is located here:
\url{https://github.com/Josh-Tang112/CMSC701_Final_Project}

\bibliographystyle{plain}
\bibliography{refs.bib}


\end{document}
