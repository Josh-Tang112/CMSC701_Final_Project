\documentclass[10pt]{article}
\usepackage{hyperref}
\title{Intermediate Progress Report:\\ Compressed Checkpoint Index}
\author{Josh Tang, Aaron Ortwein, Erik Rye}

\begin{document}
\maketitle
\section{Related work}

Kerbiriou and Chikhi examined parallel decompression of FASTQ gzip files to
speed up genomic algorithms that require FASTQ
input~\cite{kerbiriou2019parallel}. They do not build an index over the FASTQ
file by making a pass over it as we attempt to do, but rather attempt to
reconstruct data from back-references discovered by reading forward from a
random entry point.

\texttt{zlib}~\cite{roelofs2005zlib} is a C library written by Mark Adler that
is able to create and read gzip files. This library will be important for our
project as it exposes the underlying mechanics for inflating gzip files.

\texttt{pyfastx}~\cite{pyfastx} is a Python C extension that enables random
reads from FASTA and FASTQ. The functionality that it exposes is similar to our
overall goal and will be a useful source of information and point of comparison.

\texttt{indexed\_gzip}~\cite{indexedgzip} is a project by Paul McCarthy to
create an index over compressed NIFTI image files, which use gzip compression by
default. This will be another useful point of comparison in a tool that needs to
accomplish a similar task as us.



\section{Plan of Work}

\begin{itemize}

\item Determine what library to use. Initial research suggests this is zlib,
    since that's what gzip uses.
\item Determine when index entries can be made. This may require some trial and
    error creating index entries at different places, but we're not sure whether
        we can create an index entry at an arbitrary bit position, or whether it
        needs to be aligned to some boundary. 
\item Determine what information needs to be stored. This is mostly a function
    of looking at the zlib compression and decompression code, and examining the
    internal state of any exposed data structures to understand what's
        necessary. 
\item Determine what format to store the index data in. If they end up being
    large, may want to look at compressing the data or encoding it somehow.   
\item Create a library that exposes the ``read starting at sequence number
        X given an index file'' logic. This will be complete when we have a
        library we can include in another file with minimal exposed functions
        required to do parallel reads with e.g. pthreads.
\item Compare the time to read various sizes of gzipped FASTQ files using a
    number of threads with the index file and our library vs a sequential
        reader. This will be complete when we've timed the different reads and
        compared them at various file sizes and thread counts.
\end{itemize}

\section{Key Questions}

\begin{enumerate}
\item Can index entries be made at any  byte indices, or must they be created at
    a specific point in the gzip file (e.g., at a block boundary?)
\item Research suggests the zlib state requires a circular buffer of 32KB; how
    to best minimize index size?
\end{enumerate}

\section{Link to your workspace}

Our work is located here:
\url{https://github.com/Josh-Tang112/CMSC701_Final_Project}

\bibliographystyle{plain}
\bibliography{refs.bib}


\end{document}
